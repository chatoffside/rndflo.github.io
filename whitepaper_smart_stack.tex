\documentclass{article}
\usepackage[left=1.5cm, right=1.5cm, top=2cm]{geometry}

\makeatletter
\def\@seccntformat#1{%
  \expandafter\ifx\csname c@#1\endcsname\c@section\else
  \csname the#1\endcsname\quad
  \fi}
\makeatother

\usepackage{parskip}


\author{Arunabh Das \\ {das@rndflo.com}
   \and David Smith \\ {david@rndflo.com} 
   \and Steve Galati \\ {steve@rndflo.com} }

\title{SmartStack | A decentralized platform for smart contracts management and maintenance}
\date{\vspace{-5ex}}

\begin{document}

\maketitle

\section{Introduction}

The blockchain is a fundamentally revolutionary technology. Smart Contracts are the future. 

The true innovation at the heart of cryptocurrencies, the blockchain, is the technology that matters.

The blockchain is creating new opportunities in fintech for banking and financial technology companies  across the world.

For every complex question, there is one very simple solution, and it is wrong.

This is a nuanced space. This is a space that is brand new.

The blockchain represents something truly revolutionary. It is a change in our understanding of trust. A change in the way we organize
authority and trust, from hierarchical systems to network-centric flat systems. 

A re-imagining of what it means to have currency. How currency dervices its value. A new system for disintermediating intermediaries. 
A re-balancing of world affairs.

We are observing history in the making right now. In order to understand blockchain, we really need to undertand nuance. 

Blockchains enable borderless, open-global, trans-national, open system of access for financial payments and trust, that enables innovation
without permission, with high resistant to censorship, coercion, and geo-political manipulation. Blockchains uses a mathematical proof-system
that is fundamentally neutral to participants. Blockchains exhibit a principle that on the internet, we call net-neutrality and brings that to 
finance.

Finance without concern for source, destination or amount. Neutral routing of transactions, equal for everyone. Where does trust and 
authority come from?

Blockchains are introducing a fundamentally different, network-centric and flat-system that allows us to do transactions without 
recourse to authority or intermediaries and that derives trust from the collaboration and computation of thousands of nodes.


\section{Background}
Ethereum solves different problems from the problems that Bitcoin solves. Ethereum works best in co-existence with Bitcoin. 

Bitcoin provides robust security as a reserve-currency and trusted ledger with very strong immmutability and unforgeability.

Part of the reason it achieves this is by being simple in its construction, which means it can't do smart contracts.


\end{document}
